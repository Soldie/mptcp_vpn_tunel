% vim: set tw=78 sts=2 sw=2 ts=8 aw et ai:

Virtually all connection oriented communication in the Internet is done using TCP. It provides reliability, ordering and error-checking mechanisms together with the capability to adapt to network conditions in order to ensure swift and correct delivery of data between the communicating endpoints. However, TCP has been designed to use a single link and follow a single path between two IP addresses that cannot be changed without ending the layer 4 connection. Today, devices have evolved to have multiple network interfaces and the need arose for a protocol that would make full use of all the network resources in order to maximize throughput, improve reaction to failures and even optimize energy consumption.

MPTCP is the IETF's attempt to address the need mentioned above. The specification was published as an Experimental standard in RFC 6824 and a stable Linux kernel implementation is also available, although it has not been upstreamed as of the time of writing. In order to be effective as a TCP replacement, MPTCP needs to satisfy three conditions: be backwards compatible with TCP (current applications need to be able to run unmodified on MPTCP), be compatible with today's networks (run on existing devices) and perform at least as well as TCP in any given situation.

MPTCP works by establishing multiple subflows under a single socket connection, with each subflow able to take a separate path to the destination. Subflows are established using TCP SYN packets with special MPTCP options. Each subflow can be seen as a TCP flow independently. Subflows have their own sequence numbers for handling retransmissions, while there is also a per-connection sequence number mechanism to ensure reordering at the destination. MPTCP's break-before-make mechanism allows subflows to be created or destroyed while keeping the connection alive with everything being transparent to the application i.e. switching to another network interface with a different IP address.

The main obstacle for large scale deployment of MPTCP is the presence of middleboxes, intermediary routers or other network devices that interfere and modify packets that pass through them. Since MPTCP relies on newly created TCP options, this is a major issue. MPTCP researchers have found that up to 6\% of middleboxes strip unkown options from the TCP header (with that percentage increasing to 14\% when dealing with connections on port 80). Middlebox constraints have also influenced other design decisions for MPTCP, such as additional subflows in an existing connection requiring a SYN handshake because NAT devices and firewalls drop packets that are not preceded by a SYN.

Still, the major constraint for MPTCP remains performing at least as well as TCP in any conditions. Since MPTCP can use multiple types of interfaces, each with different characteristics, there is the need to understand how connection or subflow parameters can affect performance.
