% vim: set tw=78 sts=2 sw=2 ts=8 aw et ai:
The amount of virtualization in our previous testbed incurred significant overhead which we believe affected performance. We used a Ubuntu 12.04 virtual machine with software defined networking framework Mininet. Inside Mininet we defined two hosts with an Ethernet link between them. OpenVPN tunnels were created over the link. While Mininet offered useful features for link control and topology definition, the additional layer created by its process-based virtualization proved problematic.

For this report we have completely revamped our testing environment. It now consists of two separate Ubuntu 14.04 virtual machines with OpenVPN tunnels running on the single link between them. The virtualization technology is KVM, which offers improved performance and reduced overhead. Setup scripts run on both machines, creating the necessary interfaces and OpenVPN tunnels and configuring system parameters, network parameters and link bandwidth and delay limitations.

Our previous environment also suffered from constraints which prevented us from configuring different bandwidth and delay values for each tunnel over the Ethernet link. As a consequence, all our tests assumed the conditions for all tunnels were the same. In the real world, one of the usecases for link aggregation is represented by mobile devices which have at least two different network interfaces, WiFi and 3G. Since WiFi and 3G are very different in terms of bandwidth but especially delay, we need to be able to configure different link conditions in our experiments. This feature was added to the current version of the software.

The other system and link configuration methods have remained largely the same. Despite reducing the computational overhead we still eliminate encryption and message authentication. We use the sysctl interface of the Linux kernel to modify TCP and UDP buffer sizes. We deactivate inet peer storage and clear the route cache after every test. With regards to OpenVPN, we control the send and receive buffers, the transmit queue length, the MTU and we deactivate its internal fragmentation engine.
