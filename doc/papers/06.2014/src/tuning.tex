% vim: set tw=78 sts=2 sw=2 ts=8 aw et ai:

In order to maximize throughput, we have focused on the effects of varying the
send and receive buffer sizes. The buffer should be able to keep packets
depending on the TCP window size and must take into account the bandwidh and
delay of the underlying link. If a packet is received out of order, a node
must send ACKs such that the peer notices a packet was missed and wait for the
missing packet. During this time when the ACKs reach the peer and the reply
comes back, the node must keep all incoming packets pertaining to that
connection.

Based on this, we have looked at the interfaces provided by the Linux kernel
to manipulate the sizes of such buffers in order to suit our tests. We have
found several configurable values that aided us in our study, most of them
exposed via the sysctl interface.

First of all there are the limits for all types of connections. These are
configured using the net.core.rmem_default and net.core.rmem_max for receive
buffers and net.core.wmem_default and net.core.wmem_max for send buffers. The
entries ending in "default" dictate the starting size for a buffer, while the
entries with the "max" suffix provide an upper limit. We had to set these
options since they have precedence over entries which are dedicated to UDP or
TCP connections.

Secondly, we have approached the inet peer storage and the route cache. From
what we know, the inet peer storage caches TCP parameters used in previous TCP
sessions with other nodes. Since our tests change the bandwidth and delay,
such a behavior is undesirable. The parameters that influence the inet peer
storage are net.ipv4.inet_peer_maxttl and net.ipv4.inet_peer_minttl, which
control the time to live of entries, as well as net.ipv4.inet_peer_threshold,
which specifies how much space should be allocated for the entries. Setting
all of these to 0 effectively deactivates the storage. In addition, we run our
tests from high bandwidth and high delay to low bandwidth and low delay and
clear the route cache after each test, with the same goal of avoiding loss of
performance due to improper cached session parameters.

In terms of UDP traffic, there are three entries that control the buffer sizes
for UDP connections. The first one is net.ipv4.udp_mem, a vector of three
integers governing the number of pages allowed for queueing by all UDP
sockets. These three integers are the minimum, pressure and maximum values.
The only one which is not self-explanatory, the pressure value, signals that
UDP should moderate its memory consumption. In addition, the
net.ipv4.udp_rmem_min and net.ipv4.udp_wmem_min entries dictate the minimum
size, in bytes, that UDP sockets use in moderation for receiving and sending
data, respectively.

With regards to TCP traffic, we have three vector entries, each of three
values. Similar to udp_mem previously mentioned we can use net.ipv4.tcp_mem in
the same manner. Additionally, we can control the minimum, default and maximum
send buffers with the net.ipv4.tcp_wmem vector and receive buffers with the
similarly named net.ipv4.tcp_rmem.

We have also looked at solutions to control the buffers used by OpenVPN when
communicating with the tunnel devices it creates. The "--sndbuf" and
"--rcvbuf" control the sizes of the send and receive buffers over the tunnel
interfaces. Moreover, the "--txqueuelen" option controls how many packets
should be queued on the tunnel interface. We set this to 0 such that the
packets are forwarded as soon as possibly to the network interface. We have
also deactivated OpenVPN internal fragmentation engine using the "--fragment"
and "--mssfix" options. Lastly, we have changed the MTU over the tunnel
interface to larger values because this particular setting also dictates the
size of the buffers that are copied between kernel-space and user-space in
order to be forwarded to the encryption engine.

Last, but no least, we deactivate MPTCP checksumming since it would add
unnecessary computational overhead.
