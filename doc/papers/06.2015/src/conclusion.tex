% vim: set tw=78 sts=2 sw=2 ts=8 aw et ai:

The current paper has looked as a prototype solution aimed at maximizing
throughput in constrained environments, based on MPTCP and OpenVPN. We have
looked at mobile networks, since carriers employ various means of restricting
the data that passes through their network. As seen in Section \ref{sec:setup}
MPTCP filtering, per port traffic shaping and even per application traffic
shaping are common approaches.

While testing our system in terms of traffic direction, congestion control and
tunnel combination, we have achieved performance on par with normal
communication, with the added benefit of encryption, multiple paths and
circumventing some operator restrictions. We have noticed that most of our
traffic is shifted onto the UDP and DNS tunnels, which is expected given the
unfavorable interaction that can arise between TCP and MPTCP congestion
control. Additionally, we have noticed aggresive filtering of the HTTP tunnel,
hinting at deep packet inspection on the side of the carrier.

We have also looked in depth at the shaping done by the operator on DNS
tunnels. To this end, we have compared our solution with and existing DNS
tunneling solution, Iodine. The results show that the mobile operator also
does deep packet inspection on DNS packets, since Iodine places tunnel data in
the DNS payload, thus achieving improved throughput.

Although there are some issues due to carriers performing deep packet
inspection on popular ports, our prototype is a feasible solution for
enhancing application throughput and bypassing firewall restrictions. Further
research could focus on encapsulating the tunnel data in the application
payload specific to the port used, thus thwarting deep packet inspection.
Another possible improvement is the usage of dynamic tunnels, where the paths
used are not established beforehand, but created and teared down as network
conditions change.
